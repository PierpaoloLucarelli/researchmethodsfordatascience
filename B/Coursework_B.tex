\documentclass[]{article}
\usepackage{lmodern}
\usepackage{amssymb,amsmath}
\usepackage{ifxetex,ifluatex}
\usepackage{fixltx2e} % provides \textsubscript
\ifnum 0\ifxetex 1\fi\ifluatex 1\fi=0 % if pdftex
  \usepackage[T1]{fontenc}
  \usepackage[utf8]{inputenc}
\else % if luatex or xelatex
  \ifxetex
    \usepackage{mathspec}
  \else
    \usepackage{fontspec}
  \fi
  \defaultfontfeatures{Ligatures=TeX,Scale=MatchLowercase}
\fi
% use upquote if available, for straight quotes in verbatim environments
\IfFileExists{upquote.sty}{\usepackage{upquote}}{}
% use microtype if available
\IfFileExists{microtype.sty}{%
\usepackage{microtype}
\UseMicrotypeSet[protrusion]{basicmath} % disable protrusion for tt fonts
}{}
\usepackage[margin=1in]{geometry}
\usepackage{hyperref}
\hypersetup{unicode=true,
            pdftitle={Coursework Assignment B},
            pdfauthor={Navin Raj Prabhu - 4764722 , Javed Hassan - 4922573 , Pierpaolo Lucarelli - 4900626},
            pdfborder={0 0 0},
            breaklinks=true}
\urlstyle{same}  % don't use monospace font for urls
\usepackage{color}
\usepackage{fancyvrb}
\newcommand{\VerbBar}{|}
\newcommand{\VERB}{\Verb[commandchars=\\\{\}]}
\DefineVerbatimEnvironment{Highlighting}{Verbatim}{commandchars=\\\{\}}
% Add ',fontsize=\small' for more characters per line
\usepackage{framed}
\definecolor{shadecolor}{RGB}{248,248,248}
\newenvironment{Shaded}{\begin{snugshade}}{\end{snugshade}}
\newcommand{\KeywordTok}[1]{\textcolor[rgb]{0.13,0.29,0.53}{\textbf{#1}}}
\newcommand{\DataTypeTok}[1]{\textcolor[rgb]{0.13,0.29,0.53}{#1}}
\newcommand{\DecValTok}[1]{\textcolor[rgb]{0.00,0.00,0.81}{#1}}
\newcommand{\BaseNTok}[1]{\textcolor[rgb]{0.00,0.00,0.81}{#1}}
\newcommand{\FloatTok}[1]{\textcolor[rgb]{0.00,0.00,0.81}{#1}}
\newcommand{\ConstantTok}[1]{\textcolor[rgb]{0.00,0.00,0.00}{#1}}
\newcommand{\CharTok}[1]{\textcolor[rgb]{0.31,0.60,0.02}{#1}}
\newcommand{\SpecialCharTok}[1]{\textcolor[rgb]{0.00,0.00,0.00}{#1}}
\newcommand{\StringTok}[1]{\textcolor[rgb]{0.31,0.60,0.02}{#1}}
\newcommand{\VerbatimStringTok}[1]{\textcolor[rgb]{0.31,0.60,0.02}{#1}}
\newcommand{\SpecialStringTok}[1]{\textcolor[rgb]{0.31,0.60,0.02}{#1}}
\newcommand{\ImportTok}[1]{#1}
\newcommand{\CommentTok}[1]{\textcolor[rgb]{0.56,0.35,0.01}{\textit{#1}}}
\newcommand{\DocumentationTok}[1]{\textcolor[rgb]{0.56,0.35,0.01}{\textbf{\textit{#1}}}}
\newcommand{\AnnotationTok}[1]{\textcolor[rgb]{0.56,0.35,0.01}{\textbf{\textit{#1}}}}
\newcommand{\CommentVarTok}[1]{\textcolor[rgb]{0.56,0.35,0.01}{\textbf{\textit{#1}}}}
\newcommand{\OtherTok}[1]{\textcolor[rgb]{0.56,0.35,0.01}{#1}}
\newcommand{\FunctionTok}[1]{\textcolor[rgb]{0.00,0.00,0.00}{#1}}
\newcommand{\VariableTok}[1]{\textcolor[rgb]{0.00,0.00,0.00}{#1}}
\newcommand{\ControlFlowTok}[1]{\textcolor[rgb]{0.13,0.29,0.53}{\textbf{#1}}}
\newcommand{\OperatorTok}[1]{\textcolor[rgb]{0.81,0.36,0.00}{\textbf{#1}}}
\newcommand{\BuiltInTok}[1]{#1}
\newcommand{\ExtensionTok}[1]{#1}
\newcommand{\PreprocessorTok}[1]{\textcolor[rgb]{0.56,0.35,0.01}{\textit{#1}}}
\newcommand{\AttributeTok}[1]{\textcolor[rgb]{0.77,0.63,0.00}{#1}}
\newcommand{\RegionMarkerTok}[1]{#1}
\newcommand{\InformationTok}[1]{\textcolor[rgb]{0.56,0.35,0.01}{\textbf{\textit{#1}}}}
\newcommand{\WarningTok}[1]{\textcolor[rgb]{0.56,0.35,0.01}{\textbf{\textit{#1}}}}
\newcommand{\AlertTok}[1]{\textcolor[rgb]{0.94,0.16,0.16}{#1}}
\newcommand{\ErrorTok}[1]{\textcolor[rgb]{0.64,0.00,0.00}{\textbf{#1}}}
\newcommand{\NormalTok}[1]{#1}
\usepackage{graphicx,grffile}
\makeatletter
\def\maxwidth{\ifdim\Gin@nat@width>\linewidth\linewidth\else\Gin@nat@width\fi}
\def\maxheight{\ifdim\Gin@nat@height>\textheight\textheight\else\Gin@nat@height\fi}
\makeatother
% Scale images if necessary, so that they will not overflow the page
% margins by default, and it is still possible to overwrite the defaults
% using explicit options in \includegraphics[width, height, ...]{}
\setkeys{Gin}{width=\maxwidth,height=\maxheight,keepaspectratio}
\IfFileExists{parskip.sty}{%
\usepackage{parskip}
}{% else
\setlength{\parindent}{0pt}
\setlength{\parskip}{6pt plus 2pt minus 1pt}
}
\setlength{\emergencystretch}{3em}  % prevent overfull lines
\providecommand{\tightlist}{%
  \setlength{\itemsep}{0pt}\setlength{\parskip}{0pt}}
\setcounter{secnumdepth}{5}
% Redefines (sub)paragraphs to behave more like sections
\ifx\paragraph\undefined\else
\let\oldparagraph\paragraph
\renewcommand{\paragraph}[1]{\oldparagraph{#1}\mbox{}}
\fi
\ifx\subparagraph\undefined\else
\let\oldsubparagraph\subparagraph
\renewcommand{\subparagraph}[1]{\oldsubparagraph{#1}\mbox{}}
\fi

%%% Use protect on footnotes to avoid problems with footnotes in titles
\let\rmarkdownfootnote\footnote%
\def\footnote{\protect\rmarkdownfootnote}

%%% Change title format to be more compact
\usepackage{titling}

% Create subtitle command for use in maketitle
\newcommand{\subtitle}[1]{
  \posttitle{
    \begin{center}\large#1\end{center}
    }
}

\setlength{\droptitle}{-2em}

  \title{Coursework Assignment B}
    \pretitle{\vspace{\droptitle}\centering\huge}
  \posttitle{\par}
  \subtitle{CS4125 Seminar Research Methodology for Data Science}
  \author{Navin Raj Prabhu - 4764722 , Javed Hassan - 4922573 , Pierpaolo
Lucarelli - 4900626}
    \preauthor{\centering\large\emph}
  \postauthor{\par}
      \predate{\centering\large\emph}
  \postdate{\par}
    \date{4th March, 2019}


\begin{document}
\maketitle

\begin{Shaded}
\begin{Highlighting}[]
\CommentTok{#library load}
\KeywordTok{library}\NormalTok{(ggplot2)}
\KeywordTok{library}\NormalTok{(ggpubr)}
\end{Highlighting}
\end{Shaded}

\begin{verbatim}
## Loading required package: magrittr
\end{verbatim}

\begin{Shaded}
\begin{Highlighting}[]
\KeywordTok{library}\NormalTok{(car) }\CommentTok{#Package includes Levene's test }
\end{Highlighting}
\end{Shaded}

\begin{verbatim}
## Loading required package: carData
\end{verbatim}

\begin{Shaded}
\begin{Highlighting}[]
\CommentTok{#Data Proc:::}
\NormalTok{data =}\StringTok{ }\KeywordTok{read.csv}\NormalTok{(}\StringTok{"data.csv"}\NormalTok{)}

\CommentTok{# baselines dont use transfer learning}
\NormalTok{baselines <-}\StringTok{ }\NormalTok{data[data}\OperatorTok{$}\NormalTok{model }\OperatorTok{==}\StringTok{ 'B1'} \OperatorTok{|}\StringTok{ }\NormalTok{data}\OperatorTok{$}\NormalTok{model }\OperatorTok{==}\StringTok{ 'B2'} \OperatorTok{|}\StringTok{ }\NormalTok{data}\OperatorTok{$}\NormalTok{model }\OperatorTok{==}\StringTok{ 'B3'}\NormalTok{,]}

\CommentTok{# these models do use transfer learning - multiple training dataset}
\NormalTok{transferLearners <-}\StringTok{ }\NormalTok{data[data}\OperatorTok{$}\NormalTok{model }\OperatorTok{!=}\StringTok{ 'B1'} \OperatorTok{&}\StringTok{ }\NormalTok{data}\OperatorTok{$}\NormalTok{model }\OperatorTok{!=}\StringTok{ 'B2'} \OperatorTok{&}\StringTok{ }\NormalTok{data}\OperatorTok{$}\NormalTok{model }\OperatorTok{!=}\StringTok{ 'B3'} \OperatorTok{&}\StringTok{ }\NormalTok{data}\OperatorTok{$}\NormalTok{model }\OperatorTok{!=}\StringTok{ 'S'}\NormalTok{,]}

\CommentTok{# these models do use transfer learning - single training dataset}
\NormalTok{singleDataTransferLearner <-}\StringTok{ }\NormalTok{data[data}\OperatorTok{$}\NormalTok{model }\OperatorTok{==}\StringTok{ 'S'}\NormalTok{,]}


\NormalTok{singleDataTransferLearner}\OperatorTok{$}\NormalTok{train<-}\StringTok{ }\KeywordTok{ifelse}\NormalTok{(singleDataTransferLearner}\OperatorTok{$}\NormalTok{TrD1}\OperatorTok{==}\DecValTok{1}\NormalTok{, }\StringTok{'TrD1'}\NormalTok{,       }
                                  \KeywordTok{ifelse}\NormalTok{(singleDataTransferLearner}\OperatorTok{$}\NormalTok{TrD2}\OperatorTok{==}\DecValTok{1}\NormalTok{, }\StringTok{'TrD2'}\NormalTok{, }
                                  \KeywordTok{ifelse}\NormalTok{(singleDataTransferLearner}\OperatorTok{$}\NormalTok{TrD3}\OperatorTok{==}\DecValTok{1}\NormalTok{, }\StringTok{'TrD3'}\NormalTok{, }
                                  \KeywordTok{ifelse}\NormalTok{(singleDataTransferLearner}\OperatorTok{$}\NormalTok{TrD4}\OperatorTok{==}\DecValTok{1}\NormalTok{, }\StringTok{'TrD4'}\NormalTok{,}
                                  \KeywordTok{ifelse}\NormalTok{(singleDataTransferLearner}\OperatorTok{$}\NormalTok{TrD5}\OperatorTok{==}\DecValTok{1}\NormalTok{, }\StringTok{'TrD5'}\NormalTok{, }
                                  \KeywordTok{ifelse}\NormalTok{(singleDataTransferLearner}\OperatorTok{$}\NormalTok{TrD6}\OperatorTok{==}\DecValTok{1}\NormalTok{, }\StringTok{'TrD6'}\NormalTok{,}
                                  \KeywordTok{ifelse}\NormalTok{(singleDataTransferLearner}\OperatorTok{$}\NormalTok{TrD7}\OperatorTok{==}\DecValTok{1}\NormalTok{, }\StringTok{'TrD7'}\NormalTok{,}
                                  \KeywordTok{ifelse}\NormalTok{(singleDataTransferLearner}\OperatorTok{$}\NormalTok{TrD8}\OperatorTok{==}\DecValTok{1}\NormalTok{, }\StringTok{'TrD8'}\NormalTok{, }\DecValTok{1}\NormalTok{))))))))}
\end{Highlighting}
\end{Shaded}

\section{1. RQ1: How much does transfer learning improve over typical
non-transfer
learning?}\label{rq1-how-much-does-transfer-learning-improve-over-typical-non-transfer-learning}

\begin{Shaded}
\begin{Highlighting}[]
\CommentTok{#RQ1:}

\CommentTok{# plot the means}
\NormalTok{Bmean <-}\StringTok{ }\KeywordTok{mean}\NormalTok{(baselines}\OperatorTok{$}\NormalTok{score)}
\NormalTok{TLmean <-}\StringTok{ }\KeywordTok{mean}\NormalTok{(transferLearners}\OperatorTok{$}\NormalTok{score)}

\KeywordTok{hist}\NormalTok{(baselines}\OperatorTok{$}\NormalTok{score)}
\end{Highlighting}
\end{Shaded}

\includegraphics{Coursework_B_files/figure-latex/unnamed-chunk-3-1.pdf}

\begin{Shaded}
\begin{Highlighting}[]
\KeywordTok{hist}\NormalTok{(transferLearners}\OperatorTok{$}\NormalTok{score)}
\end{Highlighting}
\end{Shaded}

\includegraphics{Coursework_B_files/figure-latex/unnamed-chunk-3-2.pdf}

\begin{Shaded}
\begin{Highlighting}[]
\CommentTok{# H0: Is the mean of the transfer learners more than the mean of the baseline scores?}
\CommentTok{# we test this hypothesis using a t-test}
\NormalTok{significance <-}\StringTok{ }\KeywordTok{t.test}\NormalTok{(transferLearners}\OperatorTok{$}\NormalTok{score, baselines}\OperatorTok{$}\NormalTok{score, }\DataTypeTok{alternative=}\StringTok{"greater"}\NormalTok{)}
\NormalTok{significance}
\end{Highlighting}
\end{Shaded}

\begin{verbatim}
## 
##  Welch Two Sample t-test
## 
## data:  transferLearners$score and baselines$score
## t = 2.4239, df = 20.349, p-value = 0.01239
## alternative hypothesis: true difference in means is greater than 0
## 95 percent confidence interval:
##  0.03666383        Inf
## sample estimates:
## mean of x mean of y 
## 0.5048862 0.3780441
\end{verbatim}

conclusion - The t-test returns a p-value of 1 which means that we do
not need to reject H0. This gives an indication that the means on the
methods that do make use of transfer learning are on average better than
the ones that do not.

\section{2. RQ2: What is the effect of different strategies to
simultaneously learn one model from multiple
TrD's?}\label{rq2-what-is-the-effect-of-different-strategies-to-simultaneously-learn-one-model-from-multiple-trds}

\begin{Shaded}
\begin{Highlighting}[]
\CommentTok{#RQ2:}
\CommentTok{#Boxplot mean distribution - wrt Test Data}
\KeywordTok{ggboxplot}\NormalTok{(transferLearners,}\DataTypeTok{x =} \StringTok{"TeD"}\NormalTok{, }\DataTypeTok{y =} \StringTok{"score"}\NormalTok{, }\DataTypeTok{color =} \StringTok{"TeD"}\NormalTok{)}
\end{Highlighting}
\end{Shaded}

\includegraphics{Coursework_B_files/figure-latex/unnamed-chunk-4-1.pdf}

\begin{Shaded}
\begin{Highlighting}[]
\NormalTok{transferLearners_Bad_TeD  =}\StringTok{ }\NormalTok{transferLearners[transferLearners}\OperatorTok{$}\NormalTok{TeD }\OperatorTok{==}\StringTok{ 'TeD5'} \OperatorTok{|}\StringTok{ }\NormalTok{transferLearners}\OperatorTok{$}\NormalTok{TeD }\OperatorTok{==}\StringTok{ 'TeD7'}\NormalTok{,]}
\NormalTok{transferLearners_Good_TeD =}\StringTok{ }\NormalTok{transferLearners[transferLearners}\OperatorTok{$}\NormalTok{TeD }\OperatorTok{!=}\StringTok{ 'TeD5'} \OperatorTok{|}\StringTok{ }\NormalTok{transferLearners}\OperatorTok{$}\NormalTok{TeD }\OperatorTok{!=}\StringTok{ 'TeD7'}\NormalTok{,]}

\KeywordTok{ggboxplot}\NormalTok{(transferLearners_Bad_TeD,}\DataTypeTok{x =} \StringTok{"model"}\NormalTok{, }\DataTypeTok{y =} \StringTok{"score"}\NormalTok{, }\DataTypeTok{color =} \StringTok{"model"}\NormalTok{)}
\end{Highlighting}
\end{Shaded}

\includegraphics{Coursework_B_files/figure-latex/unnamed-chunk-4-2.pdf}

\begin{Shaded}
\begin{Highlighting}[]
\KeywordTok{ggboxplot}\NormalTok{(transferLearners_Good_TeD,}\DataTypeTok{x =} \StringTok{"model"}\NormalTok{, }\DataTypeTok{y =} \StringTok{"score"}\NormalTok{, }\DataTypeTok{color =} \StringTok{"model"}\NormalTok{)}
\end{Highlighting}
\end{Shaded}

\includegraphics{Coursework_B_files/figure-latex/unnamed-chunk-4-3.pdf}

\begin{Shaded}
\begin{Highlighting}[]
\CommentTok{#leveneTest(score ~ model, data = transferLearners)}

\CommentTok{# Compute the analysis of variance}
\NormalTok{res_aov <-}\StringTok{ }\KeywordTok{lm}\NormalTok{(score }\OperatorTok{~}\StringTok{ }\NormalTok{model }\OperatorTok{+}\StringTok{ }\NormalTok{TeD, }\DataTypeTok{data =}\NormalTok{ transferLearners)}
\CommentTok{# Summary of the analysis}
\KeywordTok{summary}\NormalTok{(res_aov)}
\end{Highlighting}
\end{Shaded}

\begin{verbatim}
## 
## Call:
## lm(formula = score ~ model + TeD, data = transferLearners)
## 
## Residuals:
##       Min        1Q    Median        3Q       Max 
## -0.253125 -0.022571  0.000525  0.037843  0.204602 
## 
## Coefficients:
##              Estimate Std. Error  t value Pr(>|t|)    
## (Intercept)  0.809621   0.003672  220.505  < 2e-16 ***
## modelM2      0.005737   0.003480    1.649  0.09933 .  
## modelM3      0.011600   0.003540    3.277  0.00106 ** 
## modelMF     -0.054194   0.003447  -15.724  < 2e-16 ***
## modelMN      0.022119   0.003566    6.203 6.43e-10 ***
## TeDTeD2     -0.224968   0.004174  -53.898  < 2e-16 ***
## TeDTeD3     -0.097859   0.004174  -23.445  < 2e-16 ***
## TeDTeD4     -0.315307   0.004174  -75.542  < 2e-16 ***
## TeDTeD5     -0.754505   0.004174 -180.766  < 2e-16 ***
## TeDTeD6     -0.186175   0.004174  -44.604  < 2e-16 ***
## TeDTeD7     -0.526284   0.004174 -126.088  < 2e-16 ***
## ---
## Signif. codes:  0 '***' 0.001 '**' 0.01 '*' 0.05 '.' 0.1 ' ' 1
## 
## Residual standard error: 0.05731 on 2628 degrees of freedom
## Multiple R-squared:  0.9474, Adjusted R-squared:  0.9472 
## F-statistic:  4736 on 10 and 2628 DF,  p-value: < 2.2e-16
\end{verbatim}

\begin{Shaded}
\begin{Highlighting}[]
\CommentTok{# Pairwise t-tests for - Effect of different Strategies}
\KeywordTok{pairwise.t.test}\NormalTok{( transferLearners}\OperatorTok{$}\NormalTok{score , transferLearners}\OperatorTok{$}\NormalTok{model)}
\end{Highlighting}
\end{Shaded}

\begin{verbatim}
## 
##  Pairwise comparisons using t tests with pooled SD 
## 
## data:  transferLearners$score and transferLearners$model 
## 
##    M1      M2      M3      MF     
## M2 1.00000 -       -       -      
## M3 1.00000 1.00000 -       -      
## MF 0.00202 0.00052 0.00015 -      
## MN 0.91323 1.00000 1.00000 7.1e-06
## 
## P value adjustment method: holm
\end{verbatim}

Conclusion:

\begin{enumerate}
\def\labelenumi{\arabic{enumi}.}
\tightlist
\item
  By plotting boxplot, we see that there is no apparent difference
  between the model's accuracy.
\item
  From the output of the Levene's Test results above, we can see that
  the p-value is greater than the significance level of 0.05. This means
  that there is evidence to suggest that the variance across groups is
  statistically significantly different. Therefore, we can assume the
  homogeneity of variances in the different treatment groups. So, we can
  use the ANOVA test. 3.In one-way ANOVA test, a significant p-value
  indicates that some of the group means are different, but we don't
  know which pairs of groups are different. It's possible to perform
  multiple pairwise-comparison, to determine if the mean difference
  between specific pairs of group are statistically significant.
\item
  From the results of the Pairwise and the Tuckey test, we can see that
  there no difference between the models. That is, given the data and
  eveidences of performance, we can say that there is no effect of the
  model strategies on the performance.
\item
  But, at the same time the model startegy MF is different from the
  other models, hence the strategy MF has an effect on the performance.
  But, from the boxplot, we can see that the mean distribution of MF
  model accuracies is less, hence we can stringly recommend against
  using strategy MF.
\end{enumerate}

\section{3. RQ3: What is the effect of the TrD's on the final model
performance?}\label{rq3-what-is-the-effect-of-the-trds-on-the-final-model-performance}

\begin{Shaded}
\begin{Highlighting}[]
\CommentTok{#RQ3:}

\CommentTok{#Boxplot mean distribution}
\KeywordTok{ggboxplot}\NormalTok{(singleDataTransferLearner,}\DataTypeTok{x =} \StringTok{"TeD"}\NormalTok{, }\DataTypeTok{y =} \StringTok{"score"}\NormalTok{, }\DataTypeTok{color =} \StringTok{"TeD"}\NormalTok{)}
\end{Highlighting}
\end{Shaded}

\includegraphics{Coursework_B_files/figure-latex/unnamed-chunk-5-1.pdf}

\begin{Shaded}
\begin{Highlighting}[]
\CommentTok{# Compute the analysis of variance}
\NormalTok{res_aov <-}\StringTok{ }\KeywordTok{lm}\NormalTok{(score }\OperatorTok{~}\StringTok{ }\NormalTok{TeD, }\DataTypeTok{data =}\NormalTok{ singleDataTransferLearner)}

\CommentTok{# Summary of the analysis}
\KeywordTok{summary}\NormalTok{(res_aov)}
\end{Highlighting}
\end{Shaded}

\begin{verbatim}
## 
## Call:
## lm(formula = score ~ TeD, data = singleDataTransferLearner)
## 
## Residuals:
##      Min       1Q   Median       3Q      Max 
## -0.41416 -0.03311  0.00204  0.03605  0.24426 
## 
## Coefficients:
##             Estimate Std. Error t value Pr(>|t|)    
## (Intercept)  0.66248    0.01157  57.265  < 2e-16 ***
## TeDTeD2     -0.11135    0.01636  -6.806 4.77e-11 ***
## TeDTeD3     -0.02571    0.01636  -1.572    0.117    
## TeDTeD4     -0.25073    0.01636 -15.325  < 2e-16 ***
## TeDTeD5     -0.61959    0.01636 -37.871  < 2e-16 ***
## TeDTeD6     -0.14965    0.01636  -9.147  < 2e-16 ***
## TeDTeD7     -0.47941    0.01636 -29.303  < 2e-16 ***
## ---
## Signif. codes:  0 '***' 0.001 '**' 0.01 '*' 0.05 '.' 0.1 ' ' 1
## 
## Residual standard error: 0.08015 on 329 degrees of freedom
## Multiple R-squared:  0.8821, Adjusted R-squared:   0.88 
## F-statistic: 410.3 on 6 and 329 DF,  p-value: < 2.2e-16
\end{verbatim}

\begin{Shaded}
\begin{Highlighting}[]
\KeywordTok{pairwise.t.test}\NormalTok{(singleDataTransferLearner}\OperatorTok{$}\NormalTok{score , singleDataTransferLearner}\OperatorTok{$}\NormalTok{train)}
\end{Highlighting}
\end{Shaded}

\begin{verbatim}
## 
##  Pairwise comparisons using t tests with pooled SD 
## 
## data:  singleDataTransferLearner$score and singleDataTransferLearner$train 
## 
##      TrD1 TrD2 TrD3 TrD4 TrD5 TrD6 TrD7
## TrD2 1.00 -    -    -    -    -    -   
## TrD3 1.00 1.00 -    -    -    -    -   
## TrD4 1.00 1.00 1.00 -    -    -    -   
## TrD5 0.23 1.00 1.00 1.00 -    -    -   
## TrD6 1.00 1.00 1.00 1.00 1.00 -    -   
## TrD7 0.57 1.00 1.00 1.00 1.00 1.00 -   
## TrD8 0.24 1.00 1.00 1.00 1.00 1.00 1.00
## 
## P value adjustment method: holm
\end{verbatim}

Conclusion: The same analysis strategy of the RQ2 was used here, but
before applying the annova model, the data was flattened to make data of
format, TrainData and equivalent scores. The post flattened dataset,
consisted of scores and its quivalent training data label. As a single
model can use multiply training data, copies of scores with respective
to all the training data used were generated. For Example, if Model M3
used TrD1 and TrD2 to give accuracy 0.5, the flattened data will consist
of two rows with (TrD1, 0.5) and (TrD2, 0.5) as row values.

\begin{enumerate}
\def\labelenumi{\arabic{enumi}.}
\tightlist
\item
  From boxplot we can notice an apparent difference in mean
  distribution.
\item
  This visuzlisation is backed up by the use of paired t-test. The
  t-test reveals that there no difference in the distribution between
  each of the groups. That means that the Training datasets have no
  effect on the performance of the final model.
\item
  One draw back of the system is that, the effect of combinations of
  TrDs cannot be explaine by the statistical analysis
\end{enumerate}


\end{document}
